
En este capítulo se describen de manera formal los objetivos que se presentaron
de manera genérica en la sección~\ref{sec:objetivos},
\textit{\nameref{sec:objetivos}}. 

Concretamente se exponen:

\begin{itemize}
\item Los \textbf{objetivos del sistema} resumen la funcionalidad genérica del
 proyecto.
\item Los \textbf{requisitos funcionales} definen las funciones del sistema
  software y sus componentes.
\item Los \textbf{requisitos de información} especifican los datos que el
  sistema necesita para llevar a cabo correctamente su funcionalidad.
\item Los \textbf{requisitos no funcionales} detallan criterios de diversas
  categorías que el software debe cumplir para garantizar un buen nivel de
  calidad más allá de su funcionamiento.
\end{itemize}

Para terminar el capítulo, se presentan diversas alternativas tecnológicas para
cubrir las necesidades surgidas en los requerimientos del sistema, detallando
las decisiones tomadas.


\section{Objetivos del sistema}

El objetivo principal del proyecto a grandes rasgos es:

\begin{center}
  
  \begin{tabularx}{\textwidth}{|c|X|}
    \hline
    Título & Visualización de recetas \\

    \hline

    Descripción & \textit{Amuse Bouche} permitirá crear y editar recetas de
    cocina, compartirlas y puntuarlas.\\

    \hline
  \end{tabularx}
  \captionof{table}{OBJ-1, objetivo del sistema 1}
\end{center}

\begin{center}
  
  \begin{tabularx}{\textwidth}{|c|X|}
    \hline
    Título & Accesibilidad en detalle de receta \\

    \hline

    Descripción & \textit{Amuse Bouche} permitirá dar órdenes y que el
    dispositivo lea automáticamente el contenido de la receta.\\

    \hline
  \end{tabularx}
  \captionof{table}{OBJ-2, objetivo del sistema 2}
\end{center}

\begin{center}
  
  \begin{tabularx}{\textwidth}{|c|X|}
    \hline
    Título & Búsqueda y filtrado \\

    \hline

    Descripción & \textit{Amuse Bouche} permitirá filtrar las recetas en función
    de multitud de parámetros, ya sean los ingredientes que contiene, el idioma o
    la categoría a la que pertenezca.\\

    \hline
  \end{tabularx}
  \captionof{table}{OBJ-3, objetivo del sistema 3}
\end{center}

\section{Catálogo de requisitos}

A continuación se exponen los requisitos del sistema a nivel funcional, de
información y no funcionales.

\subsection{Requisitos funcionales}

En este apartado se introducen los requisitos que describen las funciones del
sistema.

\subsubsection{Gestión de usuarios}

\begin{center}
  
  \begin{tabularx}{\textwidth}{|c|X|}
    \hline
    Título & Creación de cuenta de usuario \\

    \hline

    Descripción & El usuario de \textit{Amuse Bouche} deberá ser capaz de crear
    una cuenta de usuario para acceder a algunas de las funciones de
    \textit{Amuse Bouche}, proporcionando su nombre de usuario y contraseña.\\

    \hline
  \end{tabularx}
  \captionof{table}{RQF-1, requisito funcional 1: creación de cuenta de usuario}
\end{center}


\begin{center}
  
  \begin{tabularx}{\textwidth}{|c|X|}
    \hline
    Título & Edición de cuenta de usuario \\

    \hline

    Descripción & Un usuario logueado deberá ser capaz de editar sus datos
    personales y preferencias de usuario. En particular, deberá ser capaz de
    editar su dirección de correo electrónico y contraseña entre otros datos.\\

    \hline
  \end{tabularx}
  \captionof{table}{RQF-2, requisito funcional 2: edición de cuenta de usuario}
\end{center}



\subsubsection{Gestión de recetas}

\begin{center}
  
  \begin{tabularx}{\textwidth}{|c|X|}
    \hline
    Título & Obtención de recetas \\

    \hline

    Descripción & Un usuario (logueado o no) deberá ser capaz de obtener los
    datos de una o varias recetas.\\

    \hline
  \end{tabularx}
  \captionof{table}{RQF-3, req. funcional 3: obtención de receta}
\end{center}

\begin{center}
  
  \begin{tabularx}{\textwidth}{|c|X|}
    \hline
    Título & Creación de recetas \\

    \hline

    Descripción & Un usuario logueado deberá ser capaz de crear una receta,
    indicando su título, ingredientes y pasos entre otros datos.\\

    \hline
  \end{tabularx}
  \captionof{table}{RQF-4, req. funcional 4: creación de receta}
\end{center}


\begin{center}
  
  \begin{tabularx}{\textwidth}{|c|X|}
    \hline
    Título & Edición de recetas \\

    \hline

    Descripción & Un usuario logueado deberá ser capaz de editar una receta que
    previamente haya creado. En particular, deberá ser capaz de modificar
    todos sus datos.\\

    \hline
  \end{tabularx}
  \captionof{table}{RQF-5, req. funcional 5: edición de receta}
\end{center}


\begin{center}
  
  \begin{tabularx}{\textwidth}{|c|X|}
    \hline
    Título & Eliminación de receta \\

    \hline

    Descripción & Un usuario logueado deberá ser capaz de eliminar una receta. \\

    \hline
  \end{tabularx}
  \captionof{table}{RQF-6, req. funcional 6: eliminación de receta}
\end{center}


\begin{center}
  
  \begin{tabularx}{\textwidth}{|c|X|}
    \hline
    Título & Obtención de comentarios de receta \\

    \hline

    Descripción & Un usuario (logueado o no) deberá ser capaz de obtener los
    comentarios que se han hecho sobre una receta.\\

    \hline
  \end{tabularx}
  \captionof{table}{RQF-7, req. funcional 7: obtención de comentarios de receta}
\end{center}


\begin{center}
  
  \begin{tabularx}{\textwidth}{|c|X|}
    \hline
    Título & Comentar receta \\

    \hline

    Descripción & Un usuario logueado deberá ser capaz de añadir un comentario
    a cualquier receta compartida por él u otro usuario. \\

    \hline
  \end{tabularx}
  \captionof{table}{RQF-8, req. funcional 8: comentar receta}
\end{center}


\begin{center}
  
  \begin{tabularx}{\textwidth}{|c|X|}
    \hline
    Título & Valorar receta \\

    \hline

    Descripción & Un usuario logueado deberá ser capaz de valorar con una
    puntuación cualquier receta compartida por él u otro usuario. \\

    \hline
  \end{tabularx}
  \captionof{table}{RQF-9, req. funcional 9: valorar receta}
\end{center}

\begin{center}
  
  \begin{tabularx}{\textwidth}{|c|X|}
    \hline
    Título & Obtención de ingredientes genéricos \\

    \hline

    Descripción & Un usuario logueado o no deberá ser capaz de obtener un
    listado con los ingredientes categorizados en función de si son alérgenos
    o no.\\

    \hline
  \end{tabularx}
  \captionof{table}{RQF-10, req. funcional 10: obtención de ingredientes genéricos}
\end{center}

\subsection{Requisitos de información}
\label{sec:requisitos-informacion}

Seguidamente se exponen los requisitos que describen los datos necesarios que
requiere el sistema.

\begin{center}
  
  \begin{tabularx}{\textwidth}{|l|X|}
    \hline

    Título & Usuario\\

    \hline
    Datos específicos &

    \begin{itemize}
    \item Nombre de usuario
    \item Contraseña
    \item Dirección de correo electrónico
    \item Nombre
    \item Apellidos
    \item Fecha de cumpleaños
    \end{itemize}
    \\
    
    \hline
    
  \end{tabularx}
  \captionof{table}{IRQ-1, req. de información 1: usuario}
\end{center}


\begin{center}  
  \begin{tabularx}{\textwidth}{|l|X|}
    \hline

    Título & Receta\\

    \hline
    Datos específicos &

    \begin{itemize}
    \item Título
    \item Usuario
    \item Idioma
    \item Tipo de plato
    \item Dificultad
    \item Momento de creación
    \item Momento de actualización
    \item Tiempo de cocinado
    \item Imagen
    \item Puntuación total
    \item Puntuación media
    \item Usuarios que han puntuado
    \item Número de comentarios
    \item Comensales
    \item Fuente
    \item Lista de ingredientes
    \item Lista de pasos
    \item Lista de categorías
    \end{itemize}
    \\

    \hline
    
  \end{tabularx}
  \captionof{table}{IRQ-2, req. de información 2: receta}
\end{center}

\begin{center}  
  \begin{tabularx}{\textwidth}{|l|X|}
    \hline

    Título & Ingrediente de una receta\\

    \hline
    Datos específicos &

    \begin{itemize}
    \item Posición
    \item Cantidad
    \item Nombre
    \item Unidad de medida
    \end{itemize}
    \\
    
    \hline
    
  \end{tabularx}
  \captionof{table}{IRQ-3, req. de información 3: ingrediente de una receta}
\end{center}
    
\begin{center}  
  \begin{tabularx}{\textwidth}{|l|X|}
    \hline

    Título & Paso de una receta\\

    \hline
    Datos específicos &

    \begin{itemize}
    \item Posición
    \item Descripción
    \item (\textit{Opcional}) Imagen
    \item (\textit{Opcional}) Video
    \item Tiempo a cronometrar
    \end{itemize}
    \\
    
    \hline
    
  \end{tabularx}
  \captionof{table}{IRQ-4, req. de información 4: paso de una receta}
\end{center}


\begin{center}  
  \begin{tabularx}{\textwidth}{|l|X|}
    \hline

    Título & Categoría de una receta\\

    \hline
    Datos específicos &

    \begin{itemize}
    \item Nombre
    \end{itemize}
    \\
    
    \hline
    
  \end{tabularx}
  \captionof{table}{IRQ-5, req. de información 5: categoría de una receta}
\end{center}


\begin{center}  
  \begin{tabularx}{\textwidth}{|l|X|}
    \hline

    Título & Comentario de una receta\\

    \hline
    Datos específicos &

    \begin{itemize}
    \item Receta
    \item Usuario
    \item Comentario
    \item Momento de creación
    \end{itemize}
    \\
    
    \hline
    
  \end{tabularx}
  \captionof{table}{IRQ-6, req. de información 6: comentario de una receta}
\end{center}


\begin{center}  
  \begin{tabularx}{\textwidth}{|l|X|}
    \hline

    Título & Puntuación de una receta\\

    \hline
    Datos específicos &

    \begin{itemize}
    \item Receta
    \item Usuario
    \item Puntuación
    \end{itemize}
    \\
    
    \hline
    
  \end{tabularx}
  \captionof{table}{IRQ-7, req. de información 7: puntuación de una receta}
\end{center}


\begin{center}  
  \begin{tabularx}{\textwidth}{|l|X|}
    \hline

    Título & Ingrediente\\

    \hline
    Datos específicos &

    \begin{itemize}
    \item Código
    \item Lista de categorías
    \item Lista de traducciones
    \end{itemize}
    \\

    \hline
    
  \end{tabularx}
  \captionof{table}{IRQ-8, req. de información 8: ingrediente}
\end{center}

\begin{center}  
  \begin{tabularx}{\textwidth}{|l|X|}
    \hline

    Título & Categoría de un ingrediente\\

    \hline
    Datos específicos &

    \begin{itemize}
    \item Nombre
    \end{itemize}
    \\
    
    \hline
    
  \end{tabularx}
  \captionof{table}{IRQ-9, req. de información 9: categoría de un ingrediente}
\end{center}


\begin{center}  
  \begin{tabularx}{\textwidth}{|l|X|}
    \hline

    Título & Traducción de un ingrediente\\

    \hline
    Datos específicos &

    \begin{itemize}
    \item Traducción
    \item Idioma
    \item Momento de actualización
    \end{itemize}
    \\
    
    \hline
    
  \end{tabularx}
  \captionof{table}{IRQ-10, req. de información 10: traducción de un ingrediente}
\end{center}

\subsection{Requisitos no funcionales}

A continuación se indican los requisitos que especifican criterios usados para
juzgar la operación del software a través de varias categorias, más allá de su
comportamiento específico.

\subsubsection{Requisitos de seguridad}

\begin{center}
  
  \begin{tabularx}{\textwidth}{|c|X|}
    \hline
    Título & Restricción de acceso a usuarios \\

    \hline

    Descripción & Los datos de otros usuarios y las peticiones de creación,
    actualización y borrado sólo estarán disponibles para usuarios logueados.\\


    \hline
  \end{tabularx}
  \captionof{table}{NRQ-1, req. no funcional 1: restricción de acceso}
\end{center}

\begin{center}
  
  \begin{tabularx}{\textwidth}{|c|X|}
    \hline
    Título & Seguridad en las peticiones \\

    \hline

    Descripción & Las peticiones de usuarios logueados se realizarán utilizando
    \textit{tokens} de autenticación.\\


    \hline
  \end{tabularx}
  \captionof{table}{NRQ-2, req. no funcional 2: seguridad en peticiones}
\end{center}


\subsubsection{Requisitos de accesibilidad y usabilidad}

\begin{center}
  
  \begin{tabularx}{\textwidth}{|c|X|}
    \hline
    Título & Lectura automática \\

    \hline

    Descripción & La app deberá ser capaz de leer los pasos de la receta
    automáticamente.\\

    \hline
  \end{tabularx}
  \label{tab:accesibilidad}
  \captionof{table}{NRQ-3, req. no funcional 3: lectura automática}
\end{center}

\begin{center}
  
  \begin{tabularx}{\textwidth}{|c|X|}
    \hline
    Título & Captura de órdenes por voz \\

    \hline

    Descripción & La app deberá ser capaz de capturar órdenes del usuario por
    voz.\\

    \hline
  \end{tabularx}
  \label{tab:accesibilidad}
  \captionof{table}{NRQ-4, req. no funcional 4: órdenes por voz}
\end{center}


\section{Alternativas de solución}
\label{sec:alternativas-solucion}

En esta sección, se especifican las diferentes alternativas tecnológicas que
podrían permitir implementar los requerimientos del sistema, argumentando porqué
se han escogido ciertas aquellas con que se ha terminado desarrollando el
proyecto.

\subsection{Frameworks para el desarrollo de API REST}
\label{subsec:frameworks}

Dado que ya había trabajado en varias ocasiones con Python, quería optar por este
lenguaje para mejorar mi conocimiento sobre él. Además ya había desarrollado
APIs en otros lenguajes como PHP (utilizando Code Igniter), por lo que los había
descartado.

En cuanto a los frameworks de Python para desarrollar APIs REST,~\cite{api-python}
se barajaron los siguientes:
\begin{itemize}
\item \textit{Django} y \textit{Django REST Framework}.

\textit{Django} es un framework de desarrollo web libre que se basa en el patrón
de diseño conocido como \textit{modelo–vista–controlador}. Pone énfasis en la
reutilización, la conectividad y la extensibilidad de componentes, el desarrollo
rápido y el principio \textit{No te repitas}. Lo más importante de este framework
es su gran comunidad, gracias a la cual surgen grandes proyectos internos como
\textit{Django REST Framework}, un framework específico para desarrollar APIs REST.

Entre sus ventajas tenemos una documentación impecable y muy fácil de seguir, la
capacidad de mostrar la API directamente en la web con un sistemas de plantillas
automático, su flexibilidad y los diferentes métodos de autenticación ya
implementados y disponibles, entre otras.

\item \textit{Django} y \textit{Tastypie}.

Al igual que \textit{Django REST Framework}, surgió \textit{Tastypie} para
\textit{Django}. Sus características principales se basan en la construcción de
APIs a nivel básico. Esto descarta la posibilidad de escribir serializadores
propios (que como veremos más adelante, era requisito indispensable para este
proyecto, dada la complejidad de ciertos componentes de la API).

\item \textit{Flask}.

\textit{Flask} es un framework minimalista escrito en Python y basado en la
especificación WSGI de Werkzeug y el motor de plantillas Jinja2. Es un proyecto
nacido en 2010 y, por tanto, mucho menos maduro que \textit{Django} (creado en
2006). Suele ser utilizado en proyectos pequeños, con un par de funcionalidades
como mucho.

Aunque \textit{Flask} también posee subproyectos específicos para desarrollar
APIs REST (como, por ejemplo, \textit{Flask-RESTful} o \textit{Flask API}),
estos se quedaban bastante cortos para el tipo de API que se deseaba desarrollar
(que no contiene muchos objetos pero sí tiene ciertas particularidades en las
relaciones que requieren de una implementación personalizada).
\end{itemize}

Se tomó la decisión de continuar el desarrollo del proyecto utilizando
\textit{Django} y \textit{Django REST Framework}, por ser el framework más
robusto y que mejor se adaptaba a las necesidades de éste.


\subsection{Plataformas para el desarrollo de la aplicación móvil}

La aplicación móvil se podría desarrollar en cualquiera de las plataformas
móviles existentes. Sólo contemplaremos las dos que cubren la gran mayoría de
la cuota de mercado:

\begin{itemize}
\item \textit{Android}.~\cite{android}

\textit{Android} es un sistema operativo basado en el núcleo Linux. Fue diseñado
principalmente para dispositivos móviles con pantalla táctil, como teléfonos
inteligentes, tablets o tabléfonos; y también para relojes inteligentes,
televisores y automóviles. Inicialmente fue desarrollado por Android Inc.,
empresa que Google respaldó económicamente y que más tarde, en 2005, compró.
Actualmente Google es quien lidera e impulsa el desarrollo..

El desarrollo en \textit{Android} se realiza utilizando Java, un lenguaje de
programación compilado muy conocido y utulizado en muchos ámbitos, más allá del
desarrollo móvil.

Las aplicaciones se pueden lanzar en un mercado llamado Google Play Store.
Google cobra una tarifa única vitalicia de 25 dólares por desarrollador.

\item \textit{iOS}.~\cite{ios}

\textit{iOS} es un sistema operativo móvil de la multinacional Apple Inc.
Originalmente fue desarrollado para el iPhone, aunque después se ha usado en
dispositivos como el iPod touch y el iPad.

El desarrollo para iOS es bastante estricto: deben utilizarse computadoras Mac
con el sistema operativo OS X más actual (en este caso, El Capitan). La
programación se lleva a cabo utilizando Objective-C, un lenguaje derivado del C
cuyo uso a día de hoy prácticamente se limita al desarrollo en iOS, o bien, la
alternativa novedosa, Swift. Además, es obligatorio  utilizar XCode como
herramienta de desarrollo.

Las aplicaciones se pueden lanzar en la App Store, aplicación desde la que los
usuarios instalan nuevos programas. Se controla fuertemente las aplicaciones que
se envían mediante exhaustivas revisiones. Apple cobra una tarifa anual de 99
dólares.

\end{itemize}

La elección en este caso es Android. Básicamente hay dos razones principales:
la primera el interés en aprender a utilizar una tecnología con una cuota de
mercado que supera a todas las demás a nivel global, y la segunda, que la
biblioteca de lectura automática desarrollada por Google sólo existe para esta
plataforma. Han empezado a desarrollar una parecida para iOS, pero aún no es
estable y además no ha sido posible escoger esta plataforma por el coste.



\subsection{Sistemas de gestión de bases de datos}
\label{subsec:sistemas-bases-datos}

Los datos de la API estarán alojados en una base de datos, por lo que también
debemos elegir qué sistema de gestión de bases de datos utilizaremos.
\textit{Django} nos permite la utilización de varios sistemas, entre ellos:


\begin{itemize}
\item \textit{MariaDB}.~\cite{mariadb}

MariaDB es un sistema de gestión de bases de datos derivado de MySQL con
licencia GPL. Está desarrollado por Michael Widenius (fundador de MySQL) y la
comunidad de desarrolladores de software libre. Está considerado como la base
datos de código abierto más popular del mundo, y una de las más populares en
general junto a Oracle y Microsoft SQL Server, sobre todo para entornos de
desarrollo web.

Su utilidad y fiabilidad están más que demostradas. Cuenta con una arquitectura
cliente-servidor, y suele necesitar bastantes recursos.


\item \textit{PostgreSQL}.~\cite{postgresql}

PostgreSQL es un sistema de gestión de bases de datos objeto-relacional,
distribuido bajo licencia BSD y con su código fuente disponible libremente.
Es el sistema de gestión de bases de datos de código abierto más potente del
mercado y en sus últimas versiones no tiene nada que envidiarle a otras bases
de datos comerciales.

PostgreSQL utiliza un modelo cliente/servidor y usa multiprocesos en vez de
multihilos para garantizar la estabilidad del sistema. Un fallo en uno de los
procesos no afectará el resto y el sistema continuará funcionando.

Las ventajas principales de PostgreSQL son la velocidad y la replicación. 
PostgreSQL no solo es más rápido, sino que también es una de las pocas
bases de datos relacionas que también soporta NoSQL, gracias al soporte de JSON.
Además siempre ha sido estricto en cuanto a asegurar los datos son
válidos antes de insertarlos o actualizarlos.~\cite{postgresql-over-mariadb}


\item \textit{SQLite}.~\cite{sqlite}

SQLite es un sistema de gestión de bases de datos relacional compatible con ACID,
de dominio público contenida en una pequeña biblioteca escrita en C.

A diferencia de los sistema de gestión de bases de datos cliente-servidor, el
motor de SQLite no es un proceso independiente con el que el programa principal
se comunica. En lugar de eso, la biblioteca SQLite se enlaza con el programa
pasando a ser parte integral del mismo. El programa utiliza la funcionalidad de
SQLite a través de llamadas simples a subrutinas y funciones. Esto reduce la
latencia en el acceso a la base de datos, debido a que las llamadas a funciones
son más eficientes que la comunicación entre procesos. El conjunto de la base de
datos (definiciones, tablas, índices, y los propios datos), son guardados como
un sólo fichero estándar en la máquina host. Este diseño simple se logra
bloqueando todo el fichero de base de datos al principio de cada transacción.

\end{itemize}

Para la API se elige utilizar PostgreSQL por su velocidad y rendimiento, siendo
aún un sistema de bases de datos objeto-relacional. Para Android, se opta por
SQLite, por ser la norma general.


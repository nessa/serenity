
En este capítulo se describen de manera formal los objetivos que se presentaron
de manera genérica en la sección~\ref{sec:objetivos},
\textit{\nameref{sec:objetivos}}. 

Concretamente se exponen:

\begin{itemize}
\item Los \textbf{objetivos del sistema} resumen la funcionalidad genérica del
 proyecto.
\item Los \textbf{requisitos funcionales} definen las funciones del sistema
  software y sus componentes.
\item Los \textbf{requisitos de información} especifican los datos que el
  sistema necesita para llevar a cabo correctamente su funcionalidad.
\item Los \textbf{requisitos no funcionales} detallan criterios de diversas
  categorías que el software debe cumplir para garantizar un buen nivel de
  calidad más allá de su funcionamiento.
\end{itemize}

Para terminar el capítulo se presentan diversas alternativas tecnológicas para
cubrir las necesidades surgidas en los requerimientos del sistema, detallando
las decisiones tomadas.


\section{Objetivos del sistema}

El objetivo principal del proyecto a grandes rasgos es:

\begin{center}
  
  \begin{tabularx}{\textwidth}{|c|X|}
    \hline
    Título & Visualización de recetas \\

    \hline

    Descripción & Amuse Bouche permitirá crear y editar recetas de cocina,
    compartirlas y puntuarlas.\\

    \hline
  \end{tabularx}
  \captionof{table}{OBJ-1, objetivo del sistema 1}
\end{center}

\begin{center}
  
  \begin{tabularx}{\textwidth}{|c|X|}
    \hline
    Título & Accesibilidad en detalle de receta \\

    \hline

    Descripción & Amuse Bouche permitirá dar órdenes y que el dispositivo
    lea automáticamente el contenido de la receta.\\

    \hline
  \end{tabularx}
  \captionof{table}{OBJ-2, objetivo del sistema 2}
\end{center}

\begin{center}
  
  \begin{tabularx}{\textwidth}{|c|X|}
    \hline
    Título & Búsqueda y filtrado \\

    \hline

    Descripción & Amuse Bouche permitirá filtrar las recetas en función de
    multitud de parámetros, ya sean los ingredientes que contiene, el idioma o
    la categoría a la que pertenezca.\\

    \hline
  \end{tabularx}
  \captionof{table}{OBJ-3, objetivo del sistema 3}
\end{center}

\section{Catálogo de requisitos}

A continuación se exponen los requisitos del sistema a nivel funcional, de
información y no funcionales.

\subsection{Requisitos funcionales}

En esta subsección se introducen los requisitos que describen las funciones del
sistema.

\subsubsection{Gestión de usuarios}

\begin{center}
  
  \begin{tabularx}{\textwidth}{|c|X|}
    \hline
    Título & Creación de cuenta de usuario \\

    \hline

    Descripción & El usuario de Amuse Bouche deberá ser capaz de crear una
    cuenta de usuario para acceder a algunas de las funciones de Amuse Bouche,
    proporcionando su dirección de correo electrónico y contraseña.\\

    \hline
  \end{tabularx}
  \captionof{table}{RQF-1, requisito funcional 1: creación de cuenta de usuario}
\end{center}


\begin{center}
  
  \begin{tabularx}{\textwidth}{|c|X|}
    \hline
    Título & Edición de cuenta de usuario \\

    \hline

    Descripción & Un usuario logueado deberá ser capaz de editar sus datos
    personales y preferencias de usuario. En particular, deberá ser capaz de
    editar su dirección de correo electrónico y contraseña entre otros datos.\\

    \hline
  \end{tabularx}
  \captionof{table}{RQF-2, requisito funcional 2: edición de cuenta de usuario}
\end{center}



\subsubsection{Gestión de recetas}

\begin{center}
  
  \begin{tabularx}{\textwidth}{|c|X|}
    \hline
    Título & Obtención de recetas \\

    \hline

    Descripción & Un usuario (logueado o no) deberá ser capaz de obtener los
    datos de una o varias recetas.\\

    \hline
  \end{tabularx}
  \captionof{table}{RQF-3, req. funcional 3: obtención de receta}
\end{center}

\begin{center}
  
  \begin{tabularx}{\textwidth}{|c|X|}
    \hline
    Título & Creación de recetas \\

    \hline

    Descripción & Un usuario logueado deberá ser capaz de crear una receta,
    indicando su título, ingredientes y pasos entre otros datos.\\

    \hline
  \end{tabularx}
  \captionof{table}{RQF-4, req. funcional 4: creación de receta}
\end{center}


\begin{center}
  
  \begin{tabularx}{\textwidth}{|c|X|}
    \hline
    Título & Edición de recetas \\

    \hline

    Descripción & Un usuario logueado deberá ser capaz de editar una receta que
    previamente haya creado. En particular, deberá ser capaz de modificar
    todos sus datos.\\

    \hline
  \end{tabularx}
  \captionof{table}{RQF-5, req. funcional 5: edición de receta}
\end{center}


\begin{center}
  
  \begin{tabularx}{\textwidth}{|c|X|}
    \hline
    Título & Eliminación de receta \\

    \hline

    Descripción & Un usuario logueado deberá ser capaz de eliminar una receta. \\

    \hline
  \end{tabularx}
  \captionof{table}{RQF-6, req. funcional 6: eliminación de receta}
\end{center}


\begin{center}
  
  \begin{tabularx}{\textwidth}{|c|X|}
    \hline
    Título & Obtener comentarios de receta \\

    \hline

    Descripción & Un usuario (logueado o no) deberá ser capaz de obtener los
    comentarios que se han hecho sobre una receta.\\

    \hline
  \end{tabularx}
  \captionof{table}{RQF-7, req. funcional 7: obtener comentarios de receta}
\end{center}


\begin{center}
  
  \begin{tabularx}{\textwidth}{|c|X|}
    \hline
    Título & Comentar receta \\

    \hline

    Descripción & Un usuario logueado deberá ser capaz de añadir un comentario
    a cualquier receta compartida por él u otro usuario. \\

    \hline
  \end{tabularx}
  \captionof{table}{RQF-8, req. funcional 8: comentar receta}
\end{center}


\begin{center}
  
  \begin{tabularx}{\textwidth}{|c|X|}
    \hline
    Título & Valorar receta \\

    \hline

    Descripción & Un usuario logueado deberá ser capaz de valorar con una
    puntuación cualquier receta compartida por él u otro usuario. \\

    \hline
  \end{tabularx}
  \captionof{table}{RQF-9, req. funcional 9: valorar receta}
\end{center}


\subsection{Requisitos de información}
\label{sec:requisitos-informacion}

Seguidamente se exponen los requisitos que describen los datos necesarios que
requiere el sistema.

\begin{center}
  
  \begin{tabularx}{\textwidth}{|l|X|}
    \hline

    Título & Usuario\\

    \hline
    Datos específicos &

    \begin{itemize}
    \item Dirección de correo electrónico
    \item Contraseña
    \item Nombre
    \item Apellidos
    \item Fecha de cumpleaños
    \end{itemize}
    \\
    
    \hline
    
  \end{tabularx}
  \captionof{table}{IRQ-1, req. de información 1: usuario}
\end{center}


\begin{center}  
  \begin{tabularx}{\textwidth}{|l|X|}
    \hline

    Título & Receta\\

    \hline
    Datos específicos &

    \begin{itemize}
    \item Título
    \item Usuario
    \item Idioma
    \item Tipo de plato
    \item Dificultad
    \item Momento de creación
    \item Momento de actualización
    \item Tiempo de cocinado
    \item Imagen
    \item Puntuación total
    \item Puntuación media
    \item Usuarios que han puntuado
    \item Número de comentarios
    \item Comensales
    \item Fuente
    \item Lista de ingredientes
    \item Lista de pasos
    \item Lista de categorías
    \end{itemize}
    \\

    \hline
    
  \end{tabularx}
  \captionof{table}{IRQ-2, req. de información 2: receta}
\end{center}

\begin{center}  
  \begin{tabularx}{\textwidth}{|l|X|}
    \hline

    Título & Ingrediente de una receta\\

    \hline
    Datos específicos &

    \begin{itemize}
    \item Posición
    \item Cantidad
    \item Nombre
    \item Unidad de medida
    \end{itemize}
    \\
    
    \hline
    
  \end{tabularx}
  \captionof{table}{IRQ-3, req. de información 3: ingrediente de una receta}
\end{center}
    
\begin{center}  
  \begin{tabularx}{\textwidth}{|l|X|}
    \hline

    Título & Paso de una receta\\

    \hline
    Datos específicos &

    \begin{itemize}
    \item Posición
    \item Descripción
    \item (\textit{Opcional}) Imagen
    \item (\textit{Opcional}) Video
    \item Tiempo a cronometrar
    \end{itemize}
    \\
    
    \hline
    
  \end{tabularx}
  \captionof{table}{IRQ-4, req. de información 4: paso de una receta}
\end{center}

\begin{center}  
  \begin{tabularx}{\textwidth}{|l|X|}
    \hline

    Título & Comentario de una receta\\

    \hline
    Datos específicos &

    \begin{itemize}
    \item Receta
    \item Usuario
    \item Comentario
    \item Tiempo de creación
    \end{itemize}
    \\
    
    \hline
    
  \end{tabularx}
  \captionof{table}{IRQ-5, req. de información 5: comentario de una receta}
\end{center}


\subsection{Requisitos no funcionales}

A continuación se indican los requisitos que especifican criterios usados para
juzgar la operación del software a través de varias categorias, más allá de su
comportamiento específico.

\subsubsection{Requisitos de seguridad}

\begin{center}
  
  \begin{tabularx}{\textwidth}{|c|X|}
    \hline
    Título & Restricción de acceso a usuarios \\

    \hline

    Descripción & Los datos de otros usuarios y las peticiones de creación,
    actualización y borrado sólo estarán disponibles para usuarios logueados.\\


    \hline
  \end{tabularx}
  \captionof{table}{NRQ-1, req. no funcional 1: restricción de acceso}
\end{center}

\begin{center}
  
  \begin{tabularx}{\textwidth}{|c|X|}
    \hline
    Título & Seguridad en las peticiones \\

    \hline

    Descripción & Las peticiones de usuarios logueados se realizarán utilizando
    tokens de autenticación.\\


    \hline
  \end{tabularx}
  \captionof{table}{NRQ-2, req. no funcional 2: seguridad en peticiones}
\end{center}


\subsubsection{Requisitos de accesibilidad y usabilidad}

\begin{center}
  
  \begin{tabularx}{\textwidth}{|c|X|}
    \hline
    Título & Lectura automática \\

    \hline

    Descripción & La app deberá ser capaz de leer los pasos de la receta
    automáticamente.\\

    \hline
  \end{tabularx}
  \label{tab:accesibilidad}
  \captionof{table}{NRQ-3, req. no funcional 3: lectura automática}
\end{center}

\begin{center}
  
  \begin{tabularx}{\textwidth}{|c|X|}
    \hline
    Título & Captura de órdenes por voz \\

    \hline

    Descripción & La app deberá ser capaz de capturar órdenes del usuario por
    voz.\\

    \hline
  \end{tabularx}
  \label{tab:accesibilidad}
  \captionof{table}{NRQ-4, req. no funcional 4: órdenes por voz}
\end{center}


\section{Alternativas de solución}
\label{sec:alternativas-solucion}

En esta sección, se especifican las diferentes alternativas tecnológicas que
podrían permitir implementar los requerimientos del sistema, argumentando porqué
se han escogido ciertas aquellas con que se ha terminado desarrollando el
proyecto.

\subsection{Frameworks para el desarrollo de API REST}

Dado que ya había trabajado en varias ocasiones con Python, quería optar por este
lenguaje para mejorar mi conocimiento sobre él. Además ya había desarrollado
APIs en otros lenguajes como PHP (utilizando Code Igniter), por lo que los había
descartado.

En cuanto a los frameworks de Python para web,~\cite{api-python} se manejaron
los siguientes:
\begin{itemize}
\item Django y Djando REST Framework. rough community sentiment
 
\item Django y Tastypie.

\item Flask y Flask-RESTful.
\item Flask y Flask API.
\end{itemize}

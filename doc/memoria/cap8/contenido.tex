Tras la realización del proyecto se han dado unos resultados que dan lugar a
conclusiones, tanto personales como públicas, que se van a exponer en este
capítulo.

\section{Objetivos cumplidos}

Se han completado con éxito todos los objetivos presentes desde el inico del
proyecto (que ya quedaron reflejados en la sección~\ref{sec:objetivos}.
Concretamente:

\begin{itemize}
\item Se ha creado una aplicación de acceso público con la que los usuarios
  puede crear sus propias recetas y compartirlas. Está disponible en Google
  Play (falta enlace).
\item Dicha aplicación permite todos los requisitos que se elicitaron
  inicialmente: listado de recetas, compartir recetas, categorización y filtrado,
  varios idiomas y socialización.
\item El sistema se ha desarrollado de tal forma que es fácilmente ampliable,
  tanto la API como la aplicación Android.
\end{itemize}

Además de completar los objetivos funcionales, se han completado aquellos
transversales, mucho más personales. He ampliado mucho mis conocimientos en las
tecnologías utilizadas, hasta tal punto que en mi trabajo ya he podido
participar en proyectos Android gracias a la experiencia conseguida y
demostrada. También me ha servido para ampliar mis conocimientos culinarios,
llegando incluso a abrir un blog de cocina personal,\cite{noeliarcado} fuente
inicial y principal de las recetas publicadas en la aplicación.


\subsection{Limitaciones del proyecto}

Hay varias limitaciones a tener en cuenta en este proyecto. La mayoría tienen
que ver con las limitaciones del propio software actual.

La primera limitación importante es la plataforma móvil, dado que sólo se ha
podido desarrollar este sistema para Android, dado que las demás plataformas no
cuentan con una biblioteca de lectura automática y captación de órdenes por
voz adecuada o están en proceso de desarrollarla (como es el caso de iOS).

La segunda limitación es la imposibilidad de categorizar las recetas con sus
respectivos alérgenos en función de sus ingredientes, teniendo la seguridad de
que no va a fallar y obligando a depender o bien del usuario o bien de un grupo
de ``administradores generales'' que se encarguen de revisar estas
categorizaciones.

\section{Conclusiones personales}

\textit{Amuse Bouche} es un proyecto personal que ha surgido para suplir una
necesidad propia, real y compartida por muchas personas, y que ha cumplido su
objetivo, aunque todavía queda mucho por hacer. La aplicación es capaz de
solventar los problemas indicados en la sección~\ref{sec:situacion-actual}, por
lo que estoy muy satisfecha con el trabajo realizado.

\subsection{Lecciones aprendidas}

Este proyecto puede parecer algo simple si lo separamos en cada una de sus
funcionalidades por separado, pero el querer reunir todas ellas en un mismo
sistema ha sido complejo y duro, bastante más de lo que parece desde una
perspectiva externa.

Como aspecto positivo, he tenido la oportunidad de trabajar con tecnologías
actuales y bien estructuradas, con las cuales no me importaría volver a trabajar
en un futuro.

\section{Trabajo futuro}

Durante el desarrollo y la prueba de la aplicación, han surgido varios detalles
que me encantaría abarcar para mejorar la aplicación en un futuro.

\subsection{Unidades de medida dinámicas}

Una ampliación bastante sencilla sería añadir a la vista de detalle la capacidad
de poder indicar un multiplicador (puede que en función del número de comensales
indicado en los datos de la receta) para obtener la cantidad de ingredientes
necesaria para mayor o menos cantidad final de comida.


\subsection{Calendario de comidas}

Esta idea empieza a salirse bastante del marco inicial del proyecto, pero me
parece muy interesante para llegar a conseguir una comunidad más amplia.
Consiste en añadir un calendario a la aplicación, para poder indicar qué
recetas quieres hacer cada día.

Un sistema más complejo consistiría en hacer esto mismo conforme a una dieta
de un tipo concreto, pero para ello haría falta conocer las proteínas, los
hidratos y las grasas de cada ingrediente, lo cual sería una ampliación bastante
importante.


\subsection{Listado de la compra}

Junto con el calendario de comidas o utilizando un listado de recetas
seleccionadas, se podría dar la opción de generar el listado de la compra para
dichas recetas, calculando las cantidades sumando todos los ingredientes de todas
las recetas.


\subsection{Monetización}

Para asegurar la supervivencia de un proyecto suele ser necesario contar con un
\textbf{modelo de negocio} que sufrague los gastos que se generan. Podría
conseguirse esto haciendo que los usuarios registrados paguen, por ejemplo, 2€
al mes o un importe variable en función de la cantidad de recetas que vean.

La implementación de este modelo de monetización no es compleja, aunque la
gestión de pasarelas de pago y cobros recurrentes es uno de los procedimientos
más problemáticos a nivel de programación, sobre todo a la hora de verificar la
correcta ejecución del sistema.

Teniendo en cuenta el tipo de proyecto que se intenta llevar a cabo y la
apertura de todos los datos, creo que lo más apropiado sería crear la
posibilidad de realizar donaciones voluntarias para que la gente que considere
que el proyecto es interesante, aporte lo que considere.

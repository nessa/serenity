\section{Contexto social}

Los móviles y tabletas son parte habitual en la vida diaria de las personas, por
lo que cada día surgen nuevas aplicaciones para mejorar cualquiera de las
acciones que tenemos que llevar a cabo a diario. Esto incluye la cocina.

Son multitud las aplicaciones creadas tanto para los amantes de esta actividad,
como para aquellos a los que esto les aborrece y necesitan de un impulso para
manejarse mejor. Son habituales las aplicaciones con recetarios variados, aunque
la mayoría cerrados a actualizaciones externas. Menos son las que incluyen la
capacidad de permitir a sus usuarios subir y compartir sus propias recetas.

De esto se extra la necesidad de que existan aplicaciones para facilitar la
lectura y comprensión de las diferentes recetas de cocina.


\section{Motivación}
\label{sec:situacion-actual}

El impulso principal para desarrollar este proyecto es una necesidad personal
de la autora de esta memoria.

Soy amante de la cocina y llevo años mejorando mis técnicas culinarias a nivel
autodidacta. Trabajando a diario en la cocina por motivaciones personales y
totalmente voluntarias, he llegado a tropezar con ciertos inconvenientes a la
hora de cocinar.

El primero y más evidente viene a la hora de tratar de recordar una receta que
has cocinado pocas veces o al tratar de replicar una nueva receta que has
encontrado en un libro de cocina o en la red. Es perfectamente normal que, no
habiendo practicado mucho una receta, se te olviden ciertos ingredientes o
procedimientos. 

Por eso, la necesidad principal de tener una app que recoja y recopile todas
las recetas posibles, incluso las que nosotros mismos podemos añadir, resulta
importante.

Pero además, una vez tenemos nuestra receta disponible, ¿qué solemos hacer?
Pues colocar nuestro smarthphone o tableta en plena cocina. Todo va bien hasta
que llega la hora de actualizar la información o activar pantalla que se ha
apagado. Así, tendríamos que limpiarnos las manos con mucho cuidado para no
manchar ni estropear nuestros dispositivos.

Ante esto, surge una necesidad más imperante: la capacidad de que nuestro
dispositivo nos indique los pasos a seguir, incluso de que podamos darle
órdenes para poder ir paso a paso. Así surge la idea base de Amuse Bouche.


\section{Objetivos}
\label{sec:objetivos}

Los objetivos de un sistema se agrupan en dos tipos: \textbf{funcionales} y
\textbf{transversales}.

Los primeros se refieren a \textit{qué} debe hacer el sistema en sí mismo,
e inciden directamente en la experiencia del usuario y de potenciales
desarrolladores.

Los transversales son aquellos que se refieren al resultado final del sistema
y a la propia experiencia durante el desarrollo del mismo.

\subsection{Funcionales}
\begin{itemize}
\item Crear una app de acceso público que permita la visualización de recetas
  de cocina, su lectura automática y la recepción de órdenes del usuario.
\item Permitir la creación y la subida de nuevas recetas para poder compartirlas.
\item Habilitar la categorización de dichas recetas a través de sus propios
  ingredientes o según otros criterios, para facilitar su búsqueda, sobre todo
  de cara a personas alérgicas o intolerantes.
\item Permitir mostrar imágenes, videos o utilizar un cronómetro en cada paso
  de la receta.
\item Permitir que se suban las recetas en varios idiomas.
\item Permitir la socialización por medio de comentarios y votaciones sobre las
  recetas.
\end{itemize}


\subsection{Transversales}
\begin{itemize}
\item Ampliar mis conocimientos sobre desarrollo de aplicaciones móviles (en
  particular, en plataformas Android) y sobre desarrollo de una API REST.
\item Investigar y conocer más información acerca del mundo de la cocina.
\item Permitir ampliaciones y moficicaciones a futuro sobre la aplicación.
\end{itemize}


\section{Estado del arte}
\label{sec:estado-del-arte}

Se detalla a continuación el estado del arte de las funcionalidades más
habituales y de las apps de cocina más similares~\cite{mejores-apps-cocina}
a lo que se busca construir en el proyecto.

\subsection{Funcionalidades habituales}

\begin{itemize}
\item Recopilación de recetas
\item Compartir recetas
\item Categorización y filtrado
\item Lectura automática
\item Recepción de órdenes
\item Imágenes y videos
\item Cronómetro
\item Socialización: permitir votar y comentar las recetas de otros usuarios.
\item Varios idiomas
\end{itemize}

\subsection{Aps similares}

Hay muchas más aplicaciones de cocina de las que se incluyen en este apartado,
por eso solo se indican las más populares y relevantes:

\textbf{Canal Cocina} permite al usuario (ya logueado) subir sus propias
recetas, descargar las nuevas recetas automáticamente, consultar las recetas
subidas por otros usuarios, comentar y valorar las recetas y filtrar las
recetas por cocinero o programa. También incluye videos explicativos.

\textbf{Nestlé Cocina} permite consultar las recetas en español y menús variados,
planificar la dieta, filtrar y buscar recetas, marcar recetas como favoritas
y compartirlas. Hay más 2000 recetas fijas y no permite a los usuarios subir
nada.

\textbf{Recetario} posee una gran comunidad de usuarios que comparten sus
recetas entre ellos. También un modo online y offline, puede filtrar recetas
por varios criterios y contiene un cronómetro.

\textbf{¿Qué cocino hoy?} es un recetario con más de 4000 recetas fijas. Posee
un modo offline y un modo de lectura automática, que permite escuchar las recetas
mientras se cocina.

\subsection{Comparativa}

En la figura~\ref{comparativa_apps} se hace una comparativa de los requisitos
que se buscan en el proyecto, verificando si se ofrecen o no en las alternativas
de servicio previamente estudiadas.

Una carencia importante bastante generalizada es la falta de aplicaciones que
faciliten la accesibilidad a la información de las recetas (con lectura
automática) y ninguna de ellas posee ningún método de captura de órdenes.

Así pues, es evidente la dificultad de encontrar una app que aúne el mayor
número de características posibles, aumentando incluso cuando sea gratuita,
quedando manifiesta la necesidad del proyecto.


\begin{sidewaystable}
  \centering
  \begin{tabular}{|l|c|c|c|c|c|}
    \hline
    \textit{Concepto} & Canal cocina & Nestlé Cocina & Recetario & ¿Qué cocino hoy? & SiteUp\\
    \hline
    Recopilación de recetas & Sí & Sí & Sí & Sí & Sí \\
    \hline
    Compartir recetas & Sí & No & Sí & No & Sí \\
    \hline
    Categorización y filtrado & Sí & Sí & Sí & Sí & Sí \\
    \hline
    Lectura automática & No & No & No & Sí & Sí \\
    \hline
    Recepción de órdenes & No & No & No & No & Sí \\
    \hline
    Imágenes y videos & Sí & Sí & Sí & Sí & Sí \\
    \hline
    Cronómetro & No & No & Sí & No & Sí \\
    \hline
    Socialización & Sí & No & Sí & No & Sí \\
    \hline
    Varios idiomas & No & No & No & No & Sí \\
    \hline
  \end{tabular}
  \caption{Tabla comparativa de apps de cocina}
  \label{comparativa_apps}
\end{sidewaystable}


\section{Alcance}

Amuse Bouche se define como una aplicación móvil para la visualización de
recetas de cocina. Los usuarios tendrán la posibilidad de crear y compartir
sus recetas, activar un modo de lectura autómatica y dar órdenes al dispositivo
para modificar esta lectura.

También se permitirá filtrar categorizar las recetas para facilitar su búsqueda,
escribir las recetas en varios idiomas y escribir comentarios y valorarlas.


\subsection{Licencia}
El proyecto está publicado como software libre bajo la licencia
\ac{GPL} versión 3. El conjunto de bibliotecas y módulos utilizados
tienen las siguientes licencias:

\begin{itemize}
\item El intérprete del lenguaje de programación \textbf{Python} se distribuye
  bajo la licencia \ac{PSFL}, una licencia permisiva de tipo \ac{BSD} y
  compatible con la \ac{GNU} \ac{GPL}.

\item \textbf{Django}~\cite{django}, el framework de desarrollo web sobre el
  cual se ha construido la API, se distribuye bajo la licencia \textit{\ac{BSD}}.

\item El servidor web \textbf{nginx}~\cite{nginx} está licenciado bajo la
  licencia \ac{BSD} simplificada.

\item Los siguientes paquetes de Python utilizan también la licencia \ac{BSD}:
  \begin{itemize}
  \item django-rest-framework
  \end{itemize}
\end{itemize}



\section{Estructura del documento}
El presente documento se rige según la siguiente estructura:

\begin{itemize}
\item \textbf{\nameref{chap:introduccion}}. Se exponen el contexto, las
  motivaciones y los objetivos del proyecto \textbf{Amuse Bouche}, así como
  información sobre las licencias de sus componentes, glosario y estructura del
  documento.

\item \textbf{\nameref{chap:calendario}}. Se especifica la planificación del
  proyecto, la división y extensión de sus etapas y los porcentajes de esfuerzo
  durante todo el desarrollo.

\item \textbf{\nameref{chap:requisitos}}. Se formalizan los objetivos y
  requisitos planteados en la introducción más detalladamente.

\item \textbf{\nameref{chap:analisis}}. Se detalla la fase de análisis del
  sistema, explicando los requisitos funcionales del sistema y los diferentes
  casos de uso. Se muestran los diseñors en los cuales se basan las interfaces
  visuales de los diferentes sistemas.

\item \textbf{\nameref{chap:diseno}}. Se expone en detalle la etapa de diseño
  del sistema, con los diagramas de las arquitecturas lógicas de los sistemas y
  los diagramas de diseño de datos.

\item \textbf{\nameref{chap:implementacion}}. Se detallan las decisiones de
  implementación más relevantes que tuvieron lugar durante el desarrollo del
  proyecto.

\item \textbf{\nameref{chap:pruebas}}. Se describen las pruebas que se han
  llevado a cabo sobre el proyecto para garantizar su fiabilidad y estabilidad.

\item \textbf{\nameref{chap:conclusiones}}. Comentamos las conclusiones a las
  que se ha llegado tras desarrollar el proyecto.
\end{itemize}

Y los siguientes apéndices:
\begin{itemize}
\item \textbf{\nameref{chap:manual_usuario}}. Se explica cómo usar la aplicación.
\end{itemize}

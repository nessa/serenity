En este capítulo se detalla la planificación que se ha seguido durante el
desarrollo del proyecto. Se describe la metodología escogida, las iteraciones
por las que ha pasado el proyecto, la planificación temporal y la organización
del personal.

A continuación, se exponen los recursos inventariables de los que se ha hecho
uso a lo largo del desarrollo y puesta en producción del producto, haciendo una
estimación de los costes y, por último, explicando los posibles riesgos que
podrían darse.

El desarrollo se ha ajustado razonablemente bien en cuanto a la
cantidad de horas planificadas, aunque hay que especificar que por motivos
laborales y personales y dado que no se ha podido dedicar una cantidad de horas
fijas (ni al día ni a la semana) a este proyecto, estas horas no han podido ser
continuadas, por lo que hay franjas de tiempo en las que no se han podido
desarrollar nada.

\section{Metodología de desarrollo}

Para la realización del proyecto se ha utilizado un modelo de desarrollo
iterativo incremental, definiendo en cada etapa las funcionalidades a
elaborar y completando el ciclo completo de desarrollo y pruebas. 

A continuación se detallan cada una de las iteraciones por las que ha ido
pasando el proyecto.

\subsection{Primera iteración: preparación previa}
\label{sec:primera_iteracion}

Antes de abarcar el desarrollo, era necesario definir los requisitos (definidos
en el capítulo~\ref{chap:requisitos}) que tendría el sistema y dividirlos en
niveles de prioridad. Así, en las primeras etapas de desarrollo (la segunda
iteración que aparece en la sección~\ref{sec:segunda_iteracion} y la tercera que
aparece en la sección~\ref{sec:tercera_iteracion}) se abarcarían los más
necesarios e imprescindibles y se dejarían los menos prioritarios para las
siguientes (la cuarta iteración que se especifica en la
sección~\ref{sec:cuarta_iteracion} y la quinta, en la
sección~\ref{sec:quinta_iteracion}).

Una vez bien definidos los requisitos y las necesidades del proyecto, se
estudiaron y definieron los lenguajes y \textit{frameworks} a utilizar durante
el desarrollo de este. Así, dentro de esta etapa también se dedicó tiempo a
aprender de manera autodidacta cómo funcionaban dichos sistemas: el
\textit{framework} Django para el caso de la API y todo lo relacionado con
Android para la aplicación. Se desarrollaron una API muy básica de ejemplo
utilizando Django REST Framework, además de varias aplicaciones móviles básicas,
cada una centrada en elementos básicos de Android, como pueden ser la
utilización de actividades y fragmentos o pruebas de la biblioteca de lectura
automática.

Por último, se empezó a definir el diseño visual de aplicación, tanto el
logotipo como la paleta de colores y demás detalles involucrados.


\subsection{Segunda iteración: desarrollo de la base de la API}
\label{sec:segunda_iteracion}

En esta etapa, se define el modelo de datos básico a abarcar para los requisitos
más prioritarios, que principalmente se basaban en que en la aplicación móvil
fuera posible la visualización de recetas. Además, se creó el proyecto de Django
y se construyó la estructura básica de modelos, serializadores y vistas,
desarrollando la funcionalidad CRUD necesaria para obtener la información de las
recetas, sus ingredientes y pasos.

La gran mayoría de lo que se construyó en esta etapa se han mantenido hasta el
final, tan sólo siendo ampliado en cuanto a nuevos atributos o clases que fueran
necesarios.


\subsection{Tercera iteración: desarrollo de la base de la aplicación móvil}
\label{sec:tercera_iteracion}

Tras construir la base de la API, se empezó a desarrollar la aplicación Android.
Conforme a lo aprendido en la primera iteración
(sección~\ref{sec:primera_iteracion}), se vio la necesidad de construir clases
para conectarse con la API \ac{REST} y obtener y guardar información
en la base de datos, además de todas aquellas necesarias para las vistas de la
aplicación.

Se construyó la estructura del proyecto y se implementaron las clases y métodos
necesarios descritos anteriormente, así como las \textit{actividades} y
\textit{fragmentos} que forman las vistas iniciales. Estas vistas eran las
referentes a la vista inicial (el listado de recetas) y la vista de detalle de
una receta concreta.


\subsection{Cuarta iteración: ampliación de la API}
\label{sec:cuarta_iteracion}

Una vez la API y la aplicación eran estables y cumplían los requisitos más
prioritarios, se decidió ampliar la API para abarcar el resto de requisitos. Se
añadieron los atributos y métodos necesarios para permitir el filtrado de recetas
(tanto por ingredientes o categorías como por idioma o cualquier otro atributo),
lo cual era necesario para abarcar la semántica de las recetas y la
internacionalización (para poder buscar recetas por idiomas). También se añadió
la funcionalidad necesaria para permitir comentar y puntuar las recetas.


\subsection{Quinta iteración: ampliación de la aplicación móvil}
\label{sec:quinta_iteracion}

Tras ampliar la API como se indica en la sección~\ref{sec:cuarta_iteracion},
se actualizaron los métodos que trabajaban con la base de datos para añadir
los nuevos atributos. También se añadieron nuevas vistas a la aplicación móvil,
como la vista de filtrado, la vista de edición y la vista de creación, así como
la vista de \textit{login}.

Hubo que reestructurar algunas vistas, como la vista de detalle de una receta,
ya que, tras utilizarla varias veces, se percibió que era poco amigable con el
usuario.

\subsection{Sexta iteración: pruebas finales y despliegue}
\label{sec:sexta_iteracion}

Para finalizar, hubo que realizar el plan de pruebas de forma más estricta para
encontrar los \textit{bugs} que surgieran y corregirlos. Este plan se llevó a
cabo una vez realizado el despliegue de la API en un servidor en producción, para
asegurar que funcionara en las mismas condiciones en las que se encontrarán los
usuarios finales.

En esta etapa también se documentó esta memoria y se afinaron los detalles
finales.


\section{Planificación temporal}
\label{sec:planificacion}

A continuación, se expone una comparación de la cantidad de horas estimadas
frente a las utilizadas realmente para cada iteración.

\begin{table}[hbtp]
  \centering
  \begin{tabular}{|l|c|c|}
    \hline
    \textbf{Iteración} & \textbf{Tiempo estimado} & \textbf{Tiempo real} \\
    \hline
    Planificación & 56 horas & 77 horas \\
    \hline
    Aprendizaje & 224 horas & 242 horas \\
    \hline
    Diseño visual & 56 horas & 22 horas \\
    \hline
    Desarrollo API & 128 horas & 110 horas \\
    \hline
    Desarrollo app & 128 horas & 132 horas \\
    \hline
    Ampliación API & 168 horas & 88 horas \\
    \hline
    Ampliación app & 216 horas & 222 horas \\
    \hline
    Despliegue & 32 horas & 26 horas \\
    \hline
    Resolución de \textit{bugs} & 160 horas & 152 horas \\
    \hline
    Documentación & 80 horas & 65 horas \\
    \hline
    \textbf{Total} & 1249 horas & 1136 horas \\
    \hline
  \end{tabular}
  \caption{Comparación de la estimación con los tiempos reales}
  \label{tab:estimacion_tiempo}
\end{table}

Cada tarea necesitaba de la terminación por completo de la anterior, por lo que
no se han podido paralelizar tiempos.

No se muestra el diagrama de Gantt dado que, como ya se ha explicado en la
introducción del presente capítulo, no se ha podido dedicar una cantidad de horas
fijas (ni al día ni a la semana) a este proyecto, por lo que las horas no han
podido ser continuadas. Por tanto, no hay una franja temporal continua de
desarrollo.

\section{Organización}

El proyecto ha sido desarrollado en su totalidad por la autora de esta memoria,
que ha llevado a cabo las labores de análisis, diseño y desarrollo, así como de
diseño de las interfaces y \textit{branding} del proyecto.

El desarrollo del proyecto ha sido revisado y guiado de forma continua por el
tutor, Manuel Palomo Duarte.

\section{Recursos inventariables}

Los recursos de \textit{hardware} utilizados durante el desarrollo y el
despliegue del proyecto son aquellos empleados por la autora para la elaboración
del proyecto y los necesarios para la implantación y puesta en marcha del
proyecto, los cuales se definen a continuación.

\begin{itemize}
\item Como puesto de desarrollo se ha utilizado un equipo con las siguientes
  características:
  \begin{itemize}
  \item Procesador Intel Core i5 CPU M480 a 2.67GHz
  \item Gráfica Intel HD Graphics 3000
  \item Memoria 8GiB DDR3
  \item HDD 320GiB
  \item Pantalla de 15,6 pulgadas
  \end{itemize}

\item Como dispositivo para las pruebas de la aplicación móvil se ha utilizado
  un \textbf{Motorola Moto G (segunda generación)} con las siguientes
  características:

  \begin{itemize}
  \item Sistema: Android 5.1/6.0
  \item CPU: Qualcomm MSM8226 Snapdragon 400 quad-core 1.2 GHz
  \item GPU: GPU Adreno 305
  \item Memoria: 1 GB.
  \item Almacenamiento: 8GB
  \item Pantalla: 5" HD, 1280x720px
  \end{itemize}

\item Como servidor para el despliegue se ha utilizado un VPS de OVH con
  las siguientes características:
  \begin{itemize}
  \item 1 vCore
  \item 2,4 GHz
  \item 2 GiB RAM
  \item SSD 20 GiB
  \item RAID 10 local
  \end{itemize}  
\end{itemize}


\section{Costes}
\label{sec:costes}

El coste total del proyecto abarcaría por un lado el coste del personal que
lo desarrolla y mantiene y, por otro, el coste de la infraestructura necesaria
para ello.

En cuanto al personal, se requieren:
\begin{itemize}
\item Un analista que se encargue de elicitar los requisitos, analizar el
  sistema y diseñarlo. Según el convenio vigente~\cite{convenio}, su sueldo base
  es de 11,24€/hora. En total, su sueldo costaría 865,48 €.
\item Un programador senior de Python para implementar la API REST. Según el
  convenio vigente~\cite{convenio}, su sueldo base es de 7,9€/hora. En total, su
  sueldo costaría 2520,1€.
\item Un programador senior de Android para implementar la app. Según el
  convenio vigente~\cite{convenio}, su sueldo base es de 7,9€/hora. En total, su
  sueldo costaría 3389,1€.
\item Un diseñador gráfico que se encargue de toda la imagen del proyecto. Según
  el convenio vigente~\cite{convenio}, su sueldo base es de 10,99€/hora. En
  total, sueldo costaría 241,78€.
\item Un administrador de sistemas que despliegue y mantenga la API y su
  servidor. Según el convenio vigente~\cite{convenio}, su sueldo base es de
  11,24€/hora. 
\end{itemize}

Teniendo en cuenta que todos los roles son desempeñados por la autora de la
memoria, este coste no es real.

En cuanto al coste de infraestructura:
\begin{itemize}
\item El puesto de desarrollo tiene un coste de 489€.
\item El dispositivo de pruebas tiene un coste de 179€.
\item El servidor de despliegue tiene un coste de x€/mes (IVA incluido). El coste
  de un año de servicios sería de...
\end{itemize}

Tanto el puesto de desarrollo como el dispositivo de pruebas no han sido
comprados para este proyecto, por lo que son amortizables más allá de este, así
que tampoco entran del cómputo real. Aún así, el coste total suponiendo que todos
los gastos fueran reales se especifica en el cuadro~\ref{tab:costes}.

\begin{table}[hbtp]
  \centering
  \begin{tabular}[h]{|l|c|c|r|}
    \hline
    \textbf{Concepto} & \textbf{Coste} & \textbf{Cantidad} & \textbf{Coste total} \\
    \hline
    \textit{Analista} & 11,24€/hora & 77 horas & 865,48€ \\
    \hline
    \textit{Programador senior Python} & 7,9€/hora & 319 horas & 2520,1€ \\
    \hline
    \textit{Programador senior Android} & 7,9€/hora & 429 horas & 3389,1€ \\
    \hline
    \textit{Diseñador gráfico} & 10,99€/hora & 22 horas & 241,78€ \\
    \hline
    \textit{Administrador de sistemas} & 11,24€/hora &  horas & € \\
    \hline
    \textit{Puesto de desarrollo} & 489€ & 1 & 489€ \\
    \hline
    \textit{Dispositivo de pruebas} & 179€ & 1 & 179€ \\
    \hline
    \textit{Servidor de despliegue} & €/mes & 12 meses & € \\
    \hline
    \textbf{Total} & \multicolumn{3}{r|}{\textbf{7684,46€}} \\
    \hline
  \end{tabular}
  \caption{Tabla de planificación de costes}
  \label{tab:costes}
\end{table}

\section{Riesgos}
\label{sec:riesgos}


A continuación se detallan los riesgos detectados para el proyecto.

\subsection{Estabilidad del servidor de despliegue}
\label{subsec:estabilidad-servidor}

La aplicación \textit{Amuse Bouche} lee información de su API, alojada en un
único servidor, por lo que si ocurriera cualquier problema que pudiera
comprometer su estabilidad, los usuarios quedarían sin posibilidad de acceder a
las recetas conectados en red. Aunque la app seguirá funcionando y podrán ver
las que sí tengan guardadas en local, no podrán tener una experiencia completa.

Con mayor presupuesto y personal, lo lógico sería replicar el sistema en
varios servidores en distintas localizaciones, con información simétrica aunque
conectados entre sí, para poder evitar un fallo de este tipo.

\subsection{Limitación de recursos}

Además de lo comentado en la subsección~\ref{subsec:estabilidad-servidor},
el servidor de despliegue tiene unos recursos limitados, por lo que pudiera
darse el caso de que si llegara una gran afluencia de peticiones, el servidor
no diera abasto y colapsara. La solución puede pasar por hacer lo mismo que en
el caso anterior y replicar el servidor y/o mejorar los recursos del actual.

\documentclass[a4paper,11pt]{article}

% Codificación
\usepackage[spanish]{babel}
\usepackage[utf8]{inputenc}

% Fuentes
\usepackage{lmodern}
\usepackage[T1]{fontenc}
\usepackage{textcomp}


% Otros paquetes
\usepackage{tabularx}
\usepackage{array}
\usepackage{xspace}
\usepackage{varioref}
\usepackage{microtype}
\usepackage{graphicx}

% Hyperref
\usepackage{color}
\definecolor{azul}{rgb}{.1,.1,.4}
\usepackage[colorlinks=true, linkcolor=azul,citecolor=azul, filecolor=azul, menucolor=azul, pagecolor=azul, urlcolor=azul]{hyperref}

% Color
\definecolor{bg}{rgb}{0.95,0.95,0.95}

% Tamaño de la página
\setlength\parindent{0mm}
\setlength\oddsidemargin{0cm}
\setlength\textwidth{16cm}

% Información
\def\fecha{25 de junio de 2014}
% Portada
\title{Definición de proyecto\\Amuse Bouche: Aplicación Android de visualización de recetas}
\author{Noelia Sales Montes}
\date{\fecha}

\setlength{\extrarowheight}{4pt}
\setlength\parindent{0mm}

\begin{document}

\maketitle

\vspace*{1cm}

\section{Objetivo principal}

El objetivo principal de este proyecto es desarrollar una aplicación móvil para dispositivos Android para la visualización de recetas de cocina.\\

La parte básica de la aplicación permitirá visualizar dichas recetas, agregándoles un sentido semántico a los ingredientes (a través de la conexión de la aplicación con DBpedia, por ejemplo) y a la propia receta en sí (mediante categorización o datos tomados del usuario o del contexto).\\

Cada receta deberá contener los siguientes datos:
\begin{itemize}
\item Creador: usuario que ha creado la receta.
\item Ingredientes: elementos que son necesarios para realizar la receta y sus correspondientes medidas (en unidades o en medidas estándares).
\item Pasos: número indeterminado de pasos necesarios para la ejecución de la receta, los cuales a su vez podrán contener:
\begin{itemize}
\item Descripción textual del paso.
\item Imagen del paso (opcional).
\item Video del paso (opcional).
\item Tiempo necesario de espera para este paso (opcional y relacionado con la activación de un cronómetro).
\end{itemize}
\item Notas: número indeterminado de apreciaciones extra de la receta o consejos.
\item Categorías (opcional): categorizaciones diversas de la receta, ya sea en función del tipo de objetos necesarios para realizar la receta (tradicional, thermomix, olla a presión, ...) o en función de otros parámetros.
\item Puntuación: en función de lo que el resto de usuarios vote.
\item Comensales: cantidad de personas a las que se podrá alimentar con la cantidad de ingredientes especificada (opcional).
\item Idioma de la receta: puede hacerse la aplicación con una perspectiva internacional, de manera que usuario indique en qué lenguaje está escrita la receta. Por otro lado, el usuario también podrá indicar en la configuración general de la aplicación las recetas de qué lenguaje o lenguajes quiere que aparezcan en las búsquedas.
\end{itemize}

Junto con la aplicación deberá ser necesario desarrollar una API para el manejo de datos entre el servidor y la aplicación, que evidentemente deberá satisfacer los requisitos de datos especificados anteriormente.

\section{Objectivos secundarios}

En función del tiempo que requiera desarrollar completamente lo especificado en el objetivo principal, se podrá ampliar la aplicación mediante los siguientes objectivos secundarios, priorizados en función de su importancia.

\subsection{Lectura automática y reconocimiento de voz}

Incluir en la vista de visualización de recetas la posibilidad de que la aplicación lea la receta completa o un paso concreto. Junto con este objetivo se puede implementar la opción de reconocer la voz del usuario para la recepción de órdenes, de manera que el usuario pueda indicar que quiere leer un paso concreto (por ejemplo) sin necesidad de tocar el dispositivo y así no ver interrumpida su actividad.\\

Ambas opciones podrán ser activadas o desactivadas en la configuración general de la aplicación.


\subsection{Añadir recetas}

Añadir una vista para que los propios usuarios puedan incluir nuevas recetas y así compartirlas con los demás usuarios.


\subsection{Dashboard inicial}

Aunque la aplicación básica contendrá un \textit{dashboard} que sirva de conexión con las vistas internas, se puede mejorar dicha vista para incluir recomendaciones de recetas en función de diversos parámetros (ya sea las recetas mejor valoradas por el usuario, las que contengan ingredientes similares, ...).\\

Junto con este objetivo se puede incluir también una búsqueda mejorada de recetas, en función de nuevos parámetros.


\subsection{Web}

Desarrollo de una web que sirva de complemento a la aplicación, para aquellos usuarios a los que les sea más cómodo añadir las recetas a través de un ordenador, por ejemplo.

\subsection{Cálculo de calorías}

En función de los ingredientes y las cantidades de éstos, podría implementarse un sistema de cálculo aproximado de calorías de la receta completa.


\end{document}

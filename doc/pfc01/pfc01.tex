\documentclass[a4paper,11pt]{article}

% Codificación
\usepackage[spanish]{babel}
\usepackage[utf8]{inputenc}

% Fuentes
\usepackage{lmodern}
\usepackage[T1]{fontenc}
\usepackage{textcomp}


% Otros paquetes
\usepackage{tabularx}
\usepackage{array}
\usepackage{xspace}
\usepackage{varioref}
\usepackage{microtype}
\usepackage{graphicx}

% Hyperref
\usepackage{color}
\definecolor{azul}{rgb}{.1,.1,.4}
\usepackage[colorlinks=true, linkcolor=azul,citecolor=azul, filecolor=azul, menucolor=azul, pagecolor=azul, urlcolor=azul]{hyperref}

% Color
\definecolor{bg}{rgb}{0.95,0.95,0.95}

% Tamaño de la página
\setlength\parindent{0mm}
\setlength\oddsidemargin{0cm}
\setlength\textwidth{16cm}

% Información
\def\fecha{25 de junio de 2014}
% Portada
\title{Amuse Bouche\\Aplicación Android de visualización de recetas}
\author{Noelia Sales Montes}
\date{\fecha}

\setlength{\extrarowheight}{4pt}
\setlength\parindent{0mm}

\begin{document}

\maketitle

\vspace*{1cm}

El objetivo principal de este proyecto es desarrollar una aplicación móvil para dispositivos Android para la visualización de recetas de cocina.\\

La parte básica de la aplicación permitirá visualizar dichas recetas, agregándole un sentido semántico a los ingredientes y a la propia receta en sí (mediante etiquetado, categorización o datos tomados del usuario o del contexto).\\

Cada uno de los pasos de la receta podrá contener, además de la propia descripción textual, elementos gráficos (imágenes o videos) y un tiempo definido, gracias al cual se podrá activar un cronómetro de tiempo de espera.\\

Además de poder acceder a otros datos básicos referentes a la propia receta, los usuarios podrán votarlas mediante un sistema de puntuación.\\

Además se le dará un caracter internacional a la aplicación, puesto que las recetas tendrán definido el idioma en el cual están escritas. Así cada usuario podrá leer las recetas del lenguaje o lenguajes que le sean relevantes.\\

Junto con la aplicación deberá ser necesario desarrollar una API para el manejo de datos entre el servidor y la aplicación, que evidentemente deberá satisfacer los requisitos de datos especificados anteriormente.

\end{document}

% ===================================================================
% Presentación con Latex Beamer
% ===================================================================
\documentclass[10pt,xcolor=svgnames]{beamer}
% \documentclass[handout,xcolor=svgnames]{beamer} %Version imprimible
% -------------------------------------------------------------------
% Paquetes personalizados
\usepackage{paquetes}
\usepackage{colores}
\usepackage{info}
\usepackage{modo}
\usepackage{licencia}
% -------------------------------------------------------------------

% Comienza el documento
\begin{document}
% Tikz -> Imágenes
\tikzstyle{every picture}+=[remember picture]
% Entorno matemático
\everymath{\displaystyle}
% ------------------------------------------------------------------

% Título a partir del fondo especial
\begin{frame}
  \maketitle
\end{frame}

% Transparencia de índice
\begin{frame}{Índice}
  \transdissolve
  \tableofcontents
\end{frame}

\section{Introducción}

\subsection{Objetivos}

\begin{frame}{Objetivos}
  \transdissolve
  Crear una aplicación móvil de acceso público que permita la visualización
  de recetas de cocina\\

  \vspace*{1cm}

  \pause

  \textsc{Requisitos:}\\
  \begin{enumerate}
  \item<2-5> Lectura automática y captura de órdenes
  \item<3-5> Creación y subida de recetas
  \item<4-5> Categorización y filtrado (alérgenos)
  \item<5> Multiidioma
  \end{enumerate}
\end{frame}


\subsection{Motivaciones}

\begin{frame}{¿Por qué este proyecto?}
  \transdissolve
  \begin{block}{Cuando cocinamos...}
    \begin{itemize}
    \item leemos las recetas desde el dispositivo
    \item paramos para ver cada paso
    \item podemos poner en riesgo el dispositivo
    \end{itemize}
  \end{block}

  \begin{center}
    Necesidad personal
  \end{center}
\end{frame}


\begin{frame}{¿Por qué Android?}
  \transdissolve
  
  \begin{block}{Android}
    Sistema operativo basado en el núcleo Linux\\
  \end{block}
  
  \begin{itemize}
  \item Popularidad
  \item Interés personal
  \item Biblioteca \texttt{Text To Speech} desarrollada por Google
  \item Biblioteca \texttt{Speech Recognizer} desarrollada por Google
  \end{itemize}
\end{frame}


\subsection{Planificación}

\begin{frame}{Diagrama de Gantt [1/2]}
  \transdissolve
  \begin{cambiarmargen}{-2.5cm}{-2.5cm}

    \begin{table}[hbtp]
      \centering
      \begin{tabular}{|l|c|c|}
        \hline
        \textbf{Iteración} & \textbf{Tiempo estimado} & \textbf{Tiempo real} \\
        \hline
        Planificación & 56 horas & 77 horas \\
        \hline
        Aprendizaje & 224 horas & 242 horas \\
        \hline
        Diseño visual & 56 horas & 22 horas \\
        \hline
        Desarrollo API & 128 horas & 110 horas \\
        \hline
        Desarrollo app & 128 horas & 132 horas \\
        \hline
        Ampliación API & 168 horas & 88 horas \\
        \hline
        Ampliación app & 216 horas & 222 horas \\
        \hline
        Despliegue & 32 horas & 26 horas \\
        \hline
        Resolución de \textit{bugs} & 160 horas & 152 horas \\
        \hline
        Documentación & 80 horas & 65 horas \\
        \hline
        \textbf{Total} & 1249 horas & 1136 horas \\
        \hline
      \end{tabular}
      \caption{Comparación de la estimación con los tiempos reales}
      \label{tab:estimacion_tiempo}
    \end{table}
    
  \end{cambiarmargen}
\end{frame}

\section{Desarrollo}




\section{Conclusiones}

\subsection{El futuro del proyecto}

\begin{frame}{Apertura del wiki}
  \transdissolve
  
  \begin{itemize}
  \item Apertura completa: sin restricciones\\
    \then{} seguimiento de la evolución del wiki

    \vspace*{0.5cm}
  \item Versión estática de los artículos principales\\
    \then{} información fiable\\
    \then{} versión en PDF exportada a \LaTeX{}
  \end{itemize}
  
\end{frame}

\subsection{Conclusión}

\begin{frame}{Objetivos cumplidos}
  \transdissolve

  \begin{enumerate}
  \item<1-3> Aplicación de acceso público para crear y compartir recetas

    \vspace*{0.2cm}
  \item<2-3> Categorización y filtrado, multiidioma...

    \vspace*{0.2cm}
  \item<3> Lectura automática y captura de órdenes
  \end{enumerate}  
  
\end{frame}


\begin{frame}{Experiencia}
  \transdissolve

  He aprendido...
  \begin{itemize}
  \item a trabajar en plataformas Android
  \item algo más de Python (Django)
  \item a utilizar Vagrant
  \item despliegue de un proyecto real en Google Play
  \item muchas recetas nuevas
  \end{itemize}
\end{frame}  


\begin{frame}{Futuras ampliaciones}
  \transdissolve

  \begin{itemize}
  \item Unidades de medida dinámicas
    \then{} Multiplicador por comensales

    \vspace*{0.5cm}
  \item Calendario de comidas

    \vspace*{0.5cm}
  \item Listado de la compra
  \end{itemize}
\end{frame}


\begin{frame}{Bibliografía}

  \begin{thebibliography}{4}
  \bibitem{man-linux} \textit{Marko Gargenta},
    \newblock \textsc{Learning Android\\Building Applications for the Android
      Market} O'Reilly Media 2011
  \bibitem{python-admin} \textit{Norman Peitek \& Marcus Pöhls},
    \newblock \textsc{Picasso: Easy Image Loading on Android} Leanpub
  \bibitem{guia-linux} \textit{Norman Peitek \& Marcus Pöhls},
    \newblock \textsc{Retrofit: Love working with APIs on Android} Leanpub
  \end{thebibliography}
\end{frame}


\begin{frame}{Preguntas}
  \transdissolve

  \begin{center}
    \Large Gracias por su atención\\
    \vspace*{1cm}
    \Huge \textcolor{naranja}{¿Preguntas?}\\
  \end{center}
\end{frame}  

\licencia

\end{document}
